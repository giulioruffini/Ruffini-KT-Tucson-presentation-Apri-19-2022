\section{The central hypothesis}

\begin{frame}[label=ladila]{The central hypothesis in KT}
The {\bf central hypothesis} in KT is that an agent has \SEP (i.e., lives stronger, more structured experiences) to the extent it has access to {\em encompassing and compressive models} ({\em good models}) to interact with the world.  \vfill

More specifically,  {\em the event  of structured experience  arises from the act of successfully comparing  good models with data. }   \vfill

{Program characteristics deriving from  the hierarchy class it belongs to, its structure and  length, determine the properties of structured experience. }


\end{frame}

\begin{frame}[label=ladila]{The central hypothesis in KT}
 
   In algorithmic terms, {\em compressive model} refers to a simple, succinct program---the optimal model defined above is best possible scenario. Compressive models are special: to be short and accurate they need to capture structure in the data.   \vfill
    
      The term {\em encompassing} refers to the amount of data  the agent's model successfully matches information from the world, and hence to its explanatory potential.   Models need to account for all the data available  to the agent  and do so well (with little error) (i.e., with high $\mathcal M$).\vfill
    
    Models are constructed from information generated by the sensorimotor system as the agent interacts with the external world. In fact, the agent's model can account for data generated by the world and by the agent itself---i.e., include a {\em self-model}.
    
 
\end{frame}