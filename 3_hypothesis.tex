\section{The central hypothesis}

\begin{frame}[label=ladila]{The central hypothesis in KT}

\begin{exampleblock}{The central hypothesis of KT}
An agent has \SEP (i.e., living stronger, more structured experiences) to the extent it has access to {\em encompassing and compressive models} ({\em good models}) to interact with the world.  

\vspace{0.5cm}

More specifically,  {\em the event  of structured experience  arises from the act of successfully comparing  good models with data. }  \vspace{0.5cm}

Program structure determines the properties of structured experience. 
\end{exampleblock}
 \vfill

 

\end{frame}

\begin{frame}[label=ladila]{The central hypothesis in KT}
 
    {\em Compressive model} refers to a simple, succinct program---the optimal model defined above is best possible scenario. Compressive models are special: to be short and accurate they need to capture structure in the data---low $\mathcal K$.  \vfill
    
        {\em Encompassing} refers to the amount of data from the world  successfully matched, and hence   explanatory potential.  Eq., high $\mathcal M$.   %Models need to account for all the data available  to the agent  and do so with little error.
        \vfill
    
    Models are constructed from information generated by the sensorimotor system as the agent interacts with the external world: should account for data generated by the external world and by the agent itself---i.e., include a {\em self-model}.
    
 
\end{frame}