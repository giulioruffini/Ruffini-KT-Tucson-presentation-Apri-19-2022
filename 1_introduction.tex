




\section{Motivation}


\begin{frame}[label=intro]{The subjective route to KT: Experience (1P)}
	We start from the fact of {\em experience}, from the first person, subjective standpoint.  \vfill
	
	Meditation, psychedelics, religious experiences suggest that experience can be pure/primordial, free from mental constructs such as the ego.\vfill
	
	From the self-evidence of our own experience, the ``what it's like to be me'', we conclude that experience (primordial experience) does not seem to allow for or require prior causes.\vfill
	
    \begin{alertblock}{Warning:  We assume {\em there exists experience.} } 
	KT does not address the hard problem of consciousness.  
	\end{alertblock}
	

\end{frame}

\begin{frame}[label=intro2]{Structured experience (\SEP)}
 We aim to build a theory around the notion of 
{\em structured experience}---where {\em mathematics} and experience meet.  \vfill

{\bf Mathematics:}  The science of structure, order, and relation that has evolved from counting, measuring, and describing the shapes of objects \citep{gray_2010}.  \vfill

Following the phenomenological tradition, we assume that this experience is {\em structured}: at least during wakefulness, there is a spatial, temporal, and conceptual organization of our first-person experience of the world and of ourselves. \vfill

	\begin{definition}[{\em structured experience} (\SEP)]
The phenomenal structure of consciousness  encompassing both sensory qualia and the spatial, temporal, and conceptual organization of our experience  \citep{VanGulick:2016aa}. 
	\end{definition}

	

\end{frame}


\begin{frame}[label=intro3]{Scientific strategy}
This definition of \SEP can be empirically explored in reporting humans, and  we   aim to characterize it with methods generally applicable to a broad range of systems.  \vfill

  The general strategy will be  to quantify the structure of experience from first person reports in humans and attempt to explain it from state measurements (e.g., EEG or fMRI) or behavior using mechanistic insights derived from  neuroscience and mathematics.  \vfill
  
  With this knowledge at hand, we can then study \SEP in other systems (other living species or artificial agents). \vfill
  
  By providing  the bases for comparison  of the state and behavior of an artificial agent and that of an \SEP-reporting human, we can provide an educated guess about the agent’s actual experience.

\end{frame}


\begin{frame}[label=intro4]{The objective route (3P): persistence and life}
We can also start by attempting to define what {\em life} is.\vfill

What remains after the passage of eons must rightfully be called a {\em persistent pattern}.\vfill

There may be several types of such patterns (= our models). Some seem static and impervious to the world, such as protons.  \vfill

\begin{definition}[Life]
Patterns that readily interact  with the world but persist by partly capturing  structure in the world they inhabit to maintain or replicate  (homeo- and meta-homeostasis).  \end{definition}\vfill

 {\bf The connection with the first viewpoint is that, in KT, this {generalized \em life is what is capable of \SEP}}. \vfill 
 
 As part of our framework, we should study the mathematics of the emergence of life. 
\end{frame}

%%%%%%%%%%%%%%%%%%%%%%%%%%%%%%%%%%%%%


%\section{AIT and Kolmogorov complexity}

%\subsection{List styles}

% \subsubsection{Itemize}

% \begin{frame}[label=S]\frametitle{ddd} 
% 	\begin{itemize}
% 		\item Item 1
% 		\item Item 2
% 		\begin{itemize}
% 			\item Sub item 1
% 			\item Sub item 2
% 			\begin{itemize}
% 				\item Sub sub sub item 1
% 				\item Sub sub sub item 2
% 			\end{itemize}
% 			\item Sub item 3
% 		\end{itemize}
% 		\item Item 3
% 	\end{itemize}
% \end{frame}

% \subsubsection{Enumerate}

% \begin{frame}[label=enumerate]\frametitle{Enumerate sample} 
% 	\begin{enumerate}
% 		\item Item 1
% 		\item Item 2
% 		\begin{enumerate}
% 			\item Sub item 1
% 			\item Sub item 2
% 			\begin{enumerate}
% 				\item Sub sub sub item 1
% 				\item Sub sub sub item 2
% 			\end{enumerate}
% 			\item Sub item 3
% 		\end{enumerate}
% 		\item Item 3
% 	\end{enumerate}
% \end{frame}

% \subsubsection{Description}

% \begin{frame}[label=description]\frametitle{Description sample}
    
% \begin{description}
% 	\item[Term 1:] Definition 1
% 	\item[Term 2:] Definition 2
% 	\item[Term 3:] Definition 3
% \end{description}

% \end{frame}

% \subsection{Boxes styles}

% \begin{frame}[label=boxes]\frametitle{Boxes Styles}
    
% \begin{block}{Block Title}
% 	Block content
% \end{block}

% \begin{alertblock}{Alert Block Title}
% 	Alert block content
% \end{alertblock}

% \begin{exampleblock}{Example Block Title}
% 	Example block content
% \end{exampleblock}

% \end{frame}

% \subsection{Block environments}

% \begin{frame}[label=environments]\frametitle{Environments Samples}
    
% 	\begin{definition}
% 	Definition content
% 	\end{definition}
  
% 	\begin{example}
% 	Example content
% 	\end{example}

% 	\begin{proof}
% 	Proof content
% 	\end{proof}  
    
% 	\begin{theorem}
% 	Theorem content
% 	\end{theorem}

% \end{frame}


% \subsection{Math}

% \begin{frame}[label=math]\frametitle{Math}
    
% \begin{equation}
% 	V_0 = k_0 \rho \sqrt{n_1^2 - n_2^2}
% \end{equation}

% \end{frame}


% \subsection{Text environments}

% \begin{frame}[label=text]\frametitle{Text Environments}
    
% \begin{quotation}
%   Quotation environment line 1\\
%   Quotation environment line 2
% \end{quotation}
% \begin{quote}
%   Quote environment line 1\\
%   Quote environment line 2
% \end{quote}
% \begin{semiverbatim}
%   Semiverbatim environment
% \end{semiverbatim}
% \begin{verse}
%   Verse environment line 1\\
%   Verse environment line 2
% \end{verse}  

% \end{frame}

% \section{Conclusion}

% \begin{frame}[label=conclu]{Conclusion slide}
% 	That's all folks!
% \end{frame}