\section{Closing}




\begin{frame}[label=ladila]{Philosophy}

KT is most naturally viewed in the context of panpsychism, a particular, somewhat controversial version of the philosophy of consciousness.  \vfill

 Although not necessary to explore the scientific implications of the theory, the adoption of panpsychism can be  motivated by simplicity criteria \citep{Symes2022-ri}.
 
\end{frame}




\begin{frame}[label=ladila]{Ethics}

 Agenthood carries with it aspects of morality associated to the role of the objective function. We can for example conceive notions of {\em good} or {\em evil} in computational terms based on the behavior of objective functions. \vfill
 
 For example, we may say that Agent's $A$ actions are {\em evil or morally wrong} for Agent $B$ (or that Agent $A$ {\em loves} $B$) if the objective function of $A$, $O_A$, includes a function of the objective function of $B$ and it increases when O$_B$ decreases, that is $O_A(O_B)$ is decreasing or (assuming differentiability) $O_A'(O_B) <0$. Conversely, we say that Agent $A$ actions are {\em good or morally right} for Agent $B$ if $O_A'(O_B) >0$. \vfill 

Synergistic behavior emerges when love is mutual, and mutually-destructive behavior takes place in the complementary case. Note the infinite loops inherent in these concepts. 
 
\end{frame}